\documentstyle{report}
\parindent = 20pt
\parskip=10pt
\pagestyle{myheadings}
\textwidth=6in
\textheight=8.5in
\hoffset -.5in
\setlength{\topmargin}{-.3in}
\raggedbottom
\markright{Computer Modeling:
    From Sports to Spaceflight.
    From Order to Chaos.}  
\begin{document}

\appendix


\chapter{Walk--through for the software}



\section{Chaos}




\subsection{The Model of Henon and Heiles}

   The important parameter of this model is E, labelled ``energy;'' it can
take values between 0 and 1/6. At the start of the program, select
$<$Run$>$ and $<$Enter value for the energy$>$ from the Menu. Start with E =
0.01. This will show non-chaotic motion.

   A screen will now appear showing a set of axes, labelled y and y', and
a closed curve. All plotted points must lie within this curve. Using the
hotkeys you have the option of entering starting conditions using the mouse
or the keyboard. Mostly you will find the mouse to be preferable; use the
keyboard if you want to choose a point very accurately. Select $<$Mouse$>$,
click on the color of your choice, click on $<$OK$>$ then click on a starting
point, and select $<$Start$>$. Points will appear lying on closed curves. When
you have enough, select $<$Mouse$>$ and choose another starting point.

   Build as many curves as you want. Note that if you start at around
(0, 0.06), then you see a different set of curves.
The computer is following the solutions in three dimensions, y, y' and x.
When x=0, a point is plotted. Plot a set of curves, with, say, starting
point (0.06, 0). Then select $<$Menu$>$ and then $<$Run$>$ $<$See
three-dimensional motion$>$. The default starting conditions will be those that
you have just used. Select them and then $<$Start$>$. You will see a projection
of the three- dimensional motion, with points plotted where x=0. The direction
of viewing can be changed, using $<$View$>$, moving the sliders and then
$<$Accept$>$. You will eventually see the points around the curves in the y-y'
plane and a torus in the three dimensions.

   Now select a higher value for the energy: maybe E=0.3. Repeat the maps.

   Increase E gradually to 0.09, noting the changes in the patterns.
For E=0.09 you should be able to find a small region of chaos: points
will appear to fill the region randomly.

   Now increase E further, but gradually. The chaotic regions will increase
in size, and the boundaries between the chaotic and non-chaotic regions
will contain interesting structure such as systems of ``islands.'' Any time
you want, have a look at the three-dimensional motion.

   Eventually, chaos will reign almost everywhere. But you may still be
able to find small pockets of locally ``non-chaotic'' behavior.


\subsection{The Lorenz Equations}

   To start, select $<$Model$>$, $<$Lorenz$>$ from the Menu, and use the default
values of the parameters. Next, select $<$Run$>$, $<$Viewing parameters$>$. You
have six options of how you can view the solutions. Select each one in turn,
to appreciate the variety. Next, select $<$Run$>$, $<$Initial conditions$>$. The
starting values of x, y and z are not too important, since many of the
solutions approach 'attractors,' regardless of where the began. But you
may want to experiment with different maximum times, if you feel that you
are seeing too much or too little detail.

   Now alternate between $<$Plot the solution$>$ and $<$Viewing parameters$>$
until you have seen all six options. Finally, select $<$See solution in
three dimensions$>$. You will see a projection of the three--dimensional
motion in x--y--z space. Use the sliders to vary the angle of viewing, and
$<$Redraw$>$ to see the new view.

   Look at the help file under $<$Help$>$, $<$Lorenz maps$>$. Then select $<$Run$>$,
$<$Plot Lorenz map$>$. You can plot points at which a coordinate has maxumum
values; pick any of the three coordinates. In spite of the chaotic structure
that you may have observed earlier, you will see points lying on a well
defined locus.

   Most solutions end up on an 'attractor.' If you only want to see this,
then you can change the starting time for plotting using $<$Run$>$, $<$Initial
conditions$>$. The calculation begins at time t=0; the plotting will begin
at the time that you specify.

   Most work on this model has been done with parameters s=10, b=8/3, and
r varied. I suggest that you do the same. In particular, investigate the
'windows,' in which the solutions are non--chaotic. See $<$Help$>$, $<$Lorenz's
equations$>$.


\subsection{The Motion of a Periodically Forced Pendulum}

   I recommend that, before running this module, you run the program for
the simple pendulum, and become familiar with phase-plane diagrams. Also
don't experiment at random with this model; for many ranges of the
parameters there is no chaos. Start with known chaotic values and build
around them.

   Start with $<$Run$>$, $<$Enter parameters$>$, from the menu. Use numbers:
f = 1, k = 0.1, w = 1. Then try $<$Swing the pendulum$>$. Experiment with
different starting conditions and enjoy the amination.

   Next, select $<$Plot x versus time$>$. Start with the default values, but
for initial conditions choose x = 0 and x-prime = 103. When that has been
plotted, choose the hotkey $<$New start$>$ and change x-prine to 103.05.
Continue, increasing x-prime by small amounts.

   You are observing one quality of chaos: sensitivity  to small changes
in the starting conditions. Notice that, in most cases, the motion
eventually settles down to regular oscillations, after initial transient
motion.

   If the motion involves many complete circulations of the pendulum
it may be better to plot Sin(x) (which is always bounded) in place of x;
or to plot x-prime as a function of the time. These options are available
for experiment.

   Choose starting conditions that lead to as much chaos as you can manage.
Now select $<$PhasePlot$>$, $<$See phase-plane plot$>$ from the menu. Use all
default conditions, but on the following screen enter a value IDelay = 0,
so that all motion from time t = 0 is plotted.

   Next select $<$See Poincare map$>$. Here the curve that you saw in the
phase-plane is sampled once during each cycle of the forcing term. You
will see isolated dots which will approach a limiting dot corresponding
to the limiting cycle at the finish of the transient terms. Often there
is no purpose in showing the transient motion; then IDelay can be used
to eliminate an initial time equal to IDelay$*$(the period of the forcing
term). Find an appropriate value of IDelay so that in the phase-plane plot
just one closed curve is shown, and only one point appears in the Poincare
map.

   Now repeat these steps using f = 1.2, k = 0.5 and w = 2/3. Everying
remains chaotic. But the Poincare map appears to have a degree of
organization.

   Change f to f = 1.6. Repeat the steps. Now the motion is non--chaotic
(after the transient terms have gone) and periodic. The Poincare map
contains just three dots.

   Results for the last two set of parameters can be predicted from a
bifurcation diagram. In this program such a diagram follows changing
values of f, followed on the horizontal axis. For each value, a Poincare
map is found and all calculated values of x-prime are plotted on the
vertical axis. So chaotic and non--chaotic regions appear clearly, and
the transitions between the two an be observed.

   Select $<$See bifurcation diagram$>$. Use the default options to replicate
figure 5.16 on page 83 of the text. You can save time, and lose some
detail, by sampling fewer values of c.


\subsection{One--Dimensional Maps}

   Some of the maps in this program can lead rapidly to very large
numbers, causing overflow problems and meaningless--looking results.
Please consult the help files, where numerical information is provided,
and, initially, make use of the default parameters. Entering numbers
more or less at randon, will only lead to frustration. Once you have a
feel for the properties of the mappings you will be able to experiment.
Utilities that may be especially useful are the cobweb diagram and the
bifurcation diagram.

   Here we shall select the Quadratic map, and use this to walk through
some of the utilities.

   To start, select $<$Model$>$, $<$Quadratic map$>$ from the menu. Upon selection,
an input screen will appear, requesting you to enter a value for a
parameter, C. Each of the available maps contains this parameter, the
variation of which will lead to different properties, chaotic and non-chaotic.

   The input screen includes an interval for appropriate values of C and
a $<$Help$>$ utility. If you click on $<$Help$>$ you will see a screen giving
more detailed information. For a start, put C = 0.

   Select $<$Maps$>$, $<$See cobwebs$>$. Use default values. You will see a
screen including graphs of the functions x and f(x). Points where these
intersect are solutions of x = f(x), and are called 'fixed points' of the
mapping. With the hotkeys select $<$Seed$>$ to enter the starting value of
the map; enter 0.9. Then press $<$Next$>$ to see the progress of the mapping:
it will approach the lower of the fixed points. Repeat with 1.1 as the seed.

   Now press $<$CleanUp$>$ to clear the screen. Then follow maps with seeds
--0.9 and --1.1. Experiment with others. One fixed point is 'attracting'
and the other is 'repelling.'

   Choose $<$New C$>$, and enter --1. Both fixed points are now repelling.
Iterations can end in a 'limit cycle.' Next try C = --1.5. Try limits
--2 $<$ x $<$ 3 for the display. Experiment with different seeds. Either
the mapping will diverge, leaving the screen, or it will be bounded
but chaotic.

   This experiment shows some properties and characters of the mapping.
A way to combine results for different values of C in some interval is
to construct a bifurcation diagram. A single value of the seed is used.
Values of C are taken on a horizontal axis. For each value of C the
mapping is calculated and values are plotted on the vertical axis.
   (In practice, the first few values are not plotted, to eliminate
initial transient effects.) Convergence to a fixed point would result
in just one point. The limit cycle that we have just seen would produce
two points. Chaos would produce a scatter of points. So the diagram
provides a lot of information on the evolution of the mapping with
changing C.

   Select $<$Maps$>$, $<$See bifurcation diagram$>$. In the input screen look
at the $<$Help$>$ file, noting some recommended numbers for different maps.
Note that these appear as default values on the screen. You can control
the number of mapped points to be ingored and the number that is then
plotted. For now, accept all default values, and see the diagram. Study
this closely. It will give you ideas for futher experiments.

   The utility $<$Maps$>$, $<$See generating function$>$ graphs the functions
x, f(x), f(f(x)), and so on. The first two show the regular fixed points.
If we were just to work with the mapping g(x) = f(f(x)), then there are
four fixed points, as you can see. This can help illustrate the role of
bifurcation.

   The utility $<$Maps$>$, $<$See maps$>$ graphs n on the horizontal axis and
the nth iterate of the map on the vertical axis. For C = 0, you will see
one horizontal line (after a few initial iterations) corresponding to
the stable fixed point. For C = --1, you wil see two lines, corresponding
to the limit cycle. For C = --1.5, you will see obvious chaos.

   Let x(n+1) and x(n) be two successive iterations. These will generate
successive points with coordinates (x(0), x(1)), (x(1), x(2)), ...
(x(n), x(n+1)), etc. The utility $<$Maps$>$, $<$(x(n), x(n+1))$>$ treats these
as cartesian coordinates and plots them. Select this utility; keep
C = --1.5. The plot that you see appears well defined, in sharp contrast
to the chaos seen in the preceding example.

   Next run $<$See histogram$>$, still with C = -- 1.5. Accept other default
parameters. The iteration has been run 10,000 times. The horizontal axis
is divided into 400 intervals, and the height of the line above an interval
is proportional to the number of iterates lying in the interval. This
shows 'preferred' values of the iteration. It can be used to interpret
the streaks that appear among the chaos of a bifurcation diagram.

   In a chaotic system iterates starting close to one another separate,
with the distance between them increasing exponentially. This continues
until the distance is comparable to the size of the system. Run the
utility $<$See exponential separation$>$, still with C = -- 1.5. Accept other
default values of the parameters. Two iterations with slightly different
seeds have been calculated. What you see is the logarith of the
difference between these (vertial axis) with the iterate number on the
horizontal axis. While the increase is taking place, the plot of the
logarith is approximately linear, showing that the increase is exponential.
The slope of the line is related to the 'Lyapounov exponent.'

   The final utility, $<$Vary seeds$>$ is similar to a bifurcation diagram,
but C is kept fixed and the seed is gradually changed. This utility was
included principally for use with Newton's method. To run $<$Vary seeds$>$
with the quadratic map, keep C = -- 1.5. Take the range for the seeds
to be between --1 and 1 (this can be important; if any mapping becomes
large, the plotting will cease). Take the vertical range to be between
--2 and 2. In this instance, the result is certainly chaotic in appearamce,
but lacking any special features. As with the bifurcation diagram, you
can zoom in an see small regions in fuller detail.

   Now play with other mappings. In each case, learn their basic
properties before experimenting.


\subsection{Two--dimensional Maps}

   In this program you can run one of four maps that are provided,
or you can make up your own. Experimenting at random will lead to
frustration since mappings can easily become unbounded. It is safest
to find parameters that give results and then change them gradually.
This walk--through will use the first of the models.

   Select $<$Model$>$, $<$One$>$ from the menu. In the input screen select
$<$Help$>$ and look at the help screen for general information about the
model. Note that the default value of the parameter C = 1.5.

   In the screen that follows, select $<$Help$>$ from the hotkeys and examine
that help screen. Then press $<$Input$>$. You have the choice of using the
keyboard of mouse to enter starting conditions. Use the keyboard when
you need to enter precise values.

   For now, use either utility. Enter, successively, (0.1, 0.1),
(0.2, 0.2), (0.25, 0.25) and (0.3, 0.3). In each case, press $<$Start$>$
after the entry and allow time for sufficient detail to be plotted.
The final run should yield chaos. Then try (0, 0.5). Experiment some
more, building more detail.

   To see a smaller region in more detail, select $<$CleanUp$>$, then choose
options to change the scale and to use the mouse. Then click the mouse
at the LOWER LEFT-hand corner of the region, and again at the UPPER RIGHT-
hand corner. Repeat the input process in this region. (You can also use
the keyboard to change the scale. Make sure that you know in advance the
new limits for x and y.)

   Once you have sufficiently investigated the case C = 1.5, you can
decrease C to look at less chaotic cases, increase C to increase the chaos.

   Try C = 1.6, 1.7,...  Continue (maybe with smaller increments) until
all patterns have disappeared.

   Model 2 can be investigated in the same ways. Start with the default
C and then increase it gradually.

   Models 3 and 4 are not so rewarding. 3 is related to the predator-prey
model. You might vary this (changing birth or predation rates, for
instance), by using the $<$Make your own$>$ utility. Several of the two--species
and disease models in this text can be modelled by difference equations.


\subsection{Newton's Method for Two Unknowns}

   In this program we use Newton's method to solve two equations
            f(x,y) = 0   and   g(x,y) = 0.
The process generates a sequence of pairs, (x,y) which may converge
to a solution or may not converge. If the equations have several
solutions, then the initial point of the iteration will decide whether
the sequence converges, and, if it does converge, which of the
solutions will be reached.

   There are three systems of equations to choose from, or you can
make up your own.

   Enter $<$Model$>$, $<$One$>$ in the menu to select the first. Use the
default values on the input screen. Then enter $<$Run$>$, $<$Follow
iterations$>$. You have the option of selecting starting conditions
using the mouse or keyboard; the mouse is simpler, unless you want
precise values.

   The four solutions to the equations are marked by crosses. Press
$<$Mouse$>$ and click on a starting point. Points for successive iterations
will be plotted. Usually these will converge quickly. The plotting will
stop, anyway, after 20 iterations. (If you become impatient, press a key.)
As you select different starting points, it can become a game to guess
which solution will be chosen.

   This randomness is used in the utilities for generating fractal
patterns. Select $<$See fractal patterns. 1$>$. A picture will be drawn
in which each pixel (or group of pixels) is colored according to the
solution chosen, or, if there is no convergence, not colored, but
left black.

   Now you can observe the 'fractal' nature of the diagram. Mentally,
pick a small region that you want to investigate. This should contain
some chaotic appearing graphics. Using the mouse, click on the lower
left corner and then the upper right corner of the region. You will
see patterns similar in nature to what you observed before. Continue
zooming in, in this way.

   Next, select $<$See fractal patterns. 2$>$. Points on the diagram are
now colorewd according to the number of iterations before convergence,
without regard to the solution that was chosen. This provides a more
colorful picture. Again, you can zoom in and see the continuation of
the patterns in smaller and smaller regions.


\section{Predator-prey Models}

   Select $<$Model$>$, $<$Volterra$>$ from the menu. Accept the default values.
You can choose to enter starting conditions using the keyboard or the
mouse. Probably you will find the mouse to be more convenient.

   Next, select $<$Run$>$, $<$Predator-prey plot$>$. This utility will generate
plots with the prey population on the horizontal, and the predator
population on the vertical axis. This model has an equilibrium (i.e.,
values of the populations that remain constant) and this is marked by
a small circle on the screen. For the default values the equilibirium
is at (1, 1). Bear in mind that, for numerical work, the populations
will be scaled. The number 1 might, for instance, designate an actual
population of one million.

   For a start, select a starting point close to the equilibrium. Follow
the solution until you see a complete cycle.

   Now look at solutions starting further from the equilibrium. As the
range of variation increases, the populations can temporarily become
very small; but a feature of this model is that neither species ever
completely dies out.

   The predator-prey plane is useful for viewing many solutions together.
To appreciate the dependence on the time, select $<$Time plots$>$. Look at
a solution starting close to the equilibrium and then one starting further
away, noting the sharp peaks for the larger cycles.

   The utility $<$Three--D plots$>$ enables you to see the two populations
and the time plotted together.

   For many species the birthrate of the prey varies cyclically, perhaps
with the seasons. To investigate this select $<$Model$>$, $<$Cyclical growth rate$>$.

   The birthrate for the prey is given by an expression
              (A0 + A1$*$Sin(p$*$t))x,
where x is the prey population and t is the time. You can enter numbers
for A0, A1 and the frequency p. (Remember that the period of change is
P = (2.Pi)/(frequency).) For a start, take A0 = 1, A1 = 0.2 and p = 1.
Select $<$Predator-prey plot$>$. You will be given the choice: 'See continuous
curve' or 'See Poincare map.' Choose the first, and pick a starting point
close to (1, 2). You will see little apparent organization. (Note that
the $<$Cleanup$>$ option allows you to start a new solution or to remove the
current mess and continue the current solution.

   Now select $<$Time plots$>$. Select a value 60 for the length of time
and (1, 2) for the start. Now a degree of organization is much clearer.

   A better way to view results in the predator-prey plane is to plot
isolated points at intervals of the period P of the growth rate cycle.
The form a 'Poincare map.' Accordingly, select $<$Predator-prey plot$>$ and
then select $<$See Poincare maps$>$. Pick a starting point close to (1, 2).
You will see a system of 'islands.' Now experiment with different
starting points and look for different sorts of structure and for chaos.

   To see how chaos can vary, change the parameter A1 in the input screen
for cyclical growth. If you increase it, the onset of chaos will be more
rapid.


\section{Sickness and Health}


\subsection{The Spread of Disease}

   With the many different options and parameters in this program
it is best to develop it gradually.

   To start select $<$Parameters$>$, $<$Enter options$>$ from the menu. Make
sure that none of the options is selected and press OK. Use the default
values for the infection and cure rates. Now select $<$Display$>$, $<$Graph
parameters$>$, again accepting the default values. You will see a graph
of the numbers susceptible, the numbers infected and the numbers
'removed' (either dead or cured).

   Experiment with changing the values of the infection and cure rates,
and see what effects this may have on the progress of the disease.
If you select $<$Phase-plane S--I plots $>$ then you will see a graph with
S horizontal and I vertical.

   Next, using $<$Parameters$>$, $<$Enter options$>$, select vaccination. In
the input screen for the vaccination parameters, select c = 1. Then
investigate the cases k = 0, 1, 2 in turn. Compare results with no
vaccination.

   Next, select incubation only. In the incubation input screen, start
with d = 10. This results in a figure that is almost the same as that
for no incubation. But repeat with b = 5, 2, 1, 0.2. Notice the degree
to which the rate of progress of the disease is slowed.

   Next, include vaccination and incubation.

   If the disease is to be followed over a long period of time, birth
and death need to be included. In $<$Enter options$>$ select ONLY $<$Birth
and death$>$. The birth and death rates can vary for those susceptible,
infected or removed. Use the default values:

      Susceptibles: birth, 3; death, 2.4.
      Infectives:   birth, 0; death, 3.
      Immune:       birth, 0; death, 1.

   Now use $<$Time plots$>$, with the time extended to t = 10. You will
observe a periodic behavior in the spread of the disease. If you use
the S--I plots, you will see a family of closed curves.

   Finally, include a periodic rate of infection. Use the formula:
                          1 + 0.2*Sin(t);
then in the 'infection' input screen a0 = 1, a1 = 0.2 and p = 1.
Go to the S--I display. Use a horizontal range of 1 to 10 and a vertical
range of 0 to 3. Select Poincare maps. Use the mouse for starting
conditions. You will see both non--chaotic and chaotic structure.



\subsection{The Spread of Malaria}

   This is a program having a lot of parameters. Random experimentation
will only lead to results that cannot be interpreted.

   For a start, I suggest that you have mosquitos that do not spread
malaria, so that the two species do not interact ---- any bites will
be harmless. However, start with some who are infected, so that you
can invetigate recovery rates.

   Select $<$Parameters$>$, $<$Parameters for the model$>$ from the menu. In
the first input screen you will see a column for humans, on the left,
and a column for mosquitos. Not all those infected will also be
infectious. To start with, the proportion of those who are infectious
will be zero; so enter 0 in each case.

   In the model the recovery and death rates are proportional to the
numbers infected. For recovery rates, try 0.1 and 1, and for death
rates, 1 and 0.1. (Humans have better doctors!) The parameters
controlling the rate at which mosquitos bite is important; for the
present, enter 0.

   The natural growth of the human population, Ph, is logistic:
            d(Ph)/dt = a0$*$Ph$*$(Pm -- Ph),
where Pm is the maximum sustainable population and a0 is a constant
that you must enter. I suggest a0 = 0.01. If the time, t, is measured
in years, this allows reasonable growth in a period of ten years.

   The second input screen is concerned with the growth rate of the
mosquitos. This is logistic, but the parameters corresponding to
a0 and Pm, above, can vary cyclically.

   For now, avoid cyclical changes. Enter B0 = 1, (mosqitos propagate
more rapidly than humans) and B1 = 0. Then M0 = 50, M1 = 0. For the
phase constants, B2 = M2 = 0. For the period of variation leave P = 1.

   In the third input screen, enter the maximum sustainable human
population, Pm, to be 50, the initial populations to be 20, each,
and the initial numbers infected to be 10 each. Finally, enter 10 as
the time span for the plots.

   You will see that the mosquitos reach their maximum population
very rapidly. You may want to experiment with the growth parameters to
very this.

   Now its time for malaria. For the proportions of those infected
that are infectious, enter 1 and 1. For the bite rate enter 1. Keep
other parameters the same. Next enter 2 for the bite rate, and note
the significant change. Now experiment with quantities such as the
death and cure rates.

   To experiment with cyclical variation, first try B1 nonzero. B1
can be larger than B0, so that at times the natural growth is negative;
but be careful here. Then take M1 to be nonzero.

   Continue experimenting; but it is best to change one parameter at a
time, and to do so gradually.


\subsection{Zeeman's Model for the HeartBeat}

   To start select $<$Run$>$, $<$Follow contraction to systole$>$ from the
menu. You will see an S--shaped curve on the screen. Learning the role
of this curve will help in the understanding of the model. The
horizontal axis, b, represents electrochemical activity. The vertical
axis, x, represents heart muscle fiber length. The parameter for the
model is T, which represents the overall tension.

   In one half of the cycle of a heartbeat there is a contraction of
the muscles to push blood into the ventricles and then down the
arteries. The fully contracted state is called systole. The position
of this state in the figure is the star lying on the curve to the right.

   Select the hotkey $<$Tension$>$. Then move the sliders to vary the tension
and the position of systole.

   Experiment with different values. Then press $<$Solution$>$, select
the mouse, and then click at some point to select a starting position.
The path to systole will be plotted. The plotting will become slow,
since the path is approaching an equilibrium. Stop the plotting by
pressing $<$Stop$>$ and then enter another starting point. You will notice
that paths are attracted toward the ends of the S--shaped curve but
are repelled from points in the middle.

   When you have several paths, press $<$DField$>$ to see a direction
field diagram in which the arrows point along the directions of the
curves.

   Experiment further with different values of the tension and
different positions of the systole.

   Now select $<$Run$>$, $<$See complete cycles$>$. Follow the prompts to
select a value for the tension and then to locate the systole and
diastole positions. (I strongly advise using the mouse. The points
must lie on the S--shaped curve.) You will see the complete cycle.
Plotting will slow down close to the equilibrium points because of
the computations, not because of any physical cause.

   If you press $<$Tension$>$ during the plotting, then you can select
a new value for the tension, and the cycle is started without need
for reselecting systole of diastole. Press $<$Input$>$ for a complete
reselection.

   If the S--shaped curve bulges further to the right than the systole,
then the cycle cannot be completed and there is a heart attack

   Finally, select $<$Run$>$, $<$Plot and EKG$>$. The selection of tension,
systole and diastole is similar to that just used. Now you will see
the cycle, as before, but also a plot of b, the electrochemical
activity as a function of the time, the equivalent of an EKG.


\section{Sports}


\subsection{Pitching a Baseball}

   To start, select $<$Run$>$, $<$Enter parameters$>$ from the menu. You
will be prompted to enter values for the drag and lift coefficients.
Accept the default values. You might change them if, for instance, you
wanted to follow the motion of a softball.  You can also select a
regular pitch or a knuckleball. Select the regular pitch.

   The following screen (which you can access also by $<$Select display$>$)
allows you to choose which projections of the motion you want to
see. Read the definitions of the axes. For now, select to see only
the x--z plane.

   The next screen (directly accessed through $<$See pitches$>$) prompts
for initial conditions. First, have no spin, vx = 125, vy = 0 and
vz = --5ft/sec. Pitch from a height of 6 ft. Press the hotkey $<$Pitch$>$ to
 see the motion.

   Now press $<$NewPitch$>$. To see the effect of topspin, enter 100 rad/sec
for wy. Leave other parameters the same. Repeat with backspin, wy = --100.

   Next change the display so that you see all three projections. Add
a spin component wz = 200. Use $<$NewPitch$>$ to modify the starting
conditions so that the pitch will finish on the x--axis; you will need
to introduce an initial y--component of velocity, vy.

   Once again, select $<$Enter parameters$>$. This time select $<$KnuckleBall$>$.
Select all three views. The ball should rotate slowly, and should be
pitched at a moderate speed. Choose 4 rad/sec for the rotation, and
initial deflection 1.5 rad. Start with vx = 100, vy = 0, vz = 5 ft/sec.
Observe the pitch. Then change the pitch parameters, one at a time to
follow their effects.



\subsection{Driving a Golf Ball}

   The forces of drag and lift are crucial for the motion of a driven
golf ball. Start by selecting $<$Run$>$, $<$See Demonstration$>$. The object
is to show motion (a) without drag or lift (so gravity is the only
force), (b) drag but no lift, and (c) drag and lift. Use the default 
values of the drag and lift coefficients. First, use values: velocity,
200 ft/sec making 10 deg with the horizontal, backspin wy = -- 300 rad/sec.

   Notice how drag diminishes the range, but backspin increases it.
Also notice the shape of the final drive; rising in almost a straight
line, with the backspin countering the force of gravity, and falling
abruptly, with the velocity reduced by drag.

   Repeat the demonstration with backspin wy = -- 400 rad/sec.

   Now use $<$Select display$>$ to see all three views of the motion,
and select $<$See drives$>$. Use the conditions from the demonstration,
but now introduce sidespin, with, for instance, wz = 100 rad/sec.

   It can be entertaining to change the parameters for lift and drag.
If you increase the parameter for the lift by a factor of ten, you
may be surprised at the resulting motion. It would also cause comment
on a golf course.


\subsection{Serving at Tennis}

   The items under $<$Run$>$ in the menu consist of: $<$Enter parameters$>$,
where you can enter values for the drag and lift coefficients; $<$Bounce
parameters$>$, where you can enter the parameter controlling the height
of a bounce and the extent to which spin will influence its direction;
$<$Select display$>$, where you can choose any or all of three views for
observing the motion; $<$See serves$>$, for entering starting conditions
and actually serving.

   If you start with $<$Run$>$, $<$Enter parameters$>$, then the remaining
prompts will automatically follow. Do this. Accept default values;
but in designing the actual serve, you can make your own values.
Start with no spin. Pick a height and location for the serve, a speed
and a horizontal direction. Then vary the angle that the serve makes
with the vertical.

   If you pick all the default values, you should see a serve that
clears the net easily and just lands in court. That serve started
with velocity making -- 1 deg with the horizontal. Reduce this starting
angle until the serve just clears the net. The difference between the
two angles shows the narrow margin of error available to the server.
Design your own serve and find the corresponding margin. Find the
effects on this margin when the speed of the serve is increased.

   Now introduce topspin, wy = 50 rad/sec, say. This should increase
the margin of error.

   Also investigate the effects of spin on bounce of the ball.
Experiment with the bounce parameters and the speed of the serve, as well
as the spin.



\subsection{Bowling a Cricket Ball}

   Under the menu item $<$Run$>$, $<$Enter Parameters$>$ enables you to
enter perameters for drag and lift; you can also select a bowl
with its trajectory and bounce dominated by spin, or a 'swinging'
bowl where there is no spin. $<$Bounce parameters$>$ enables you to
control the height of the bounce and the extent to which the spinning
ball 'grips' the ground. $<$Select display$>$ enables you to select any
or all of three views for seeing the motion. $<$See bowls$>$ enables you
to enter the starting conditions, and to see the motion.

   To start, select $<$Run$>$, $<$Enter parameters$>$ from the menu. Select
$<$Regular$>$ for the bowl. Use default values for the first three input
screens. For the starting conditions take vx = 60, vy = vz = 0 ft/sec;
wx = --50, wy = wz = 0 rad/sec. Initial height 8 ft. This is a
slow ball with the spin causing a deviation in the direction at the bounce.

   You will probably find it best to select $<$HideBall$>$ with the hotkeys.
In the display the yellow line is the wicket. This is a target for the
bowler.

   Now press $<$NewBall$>$ and change vy to --1 ft/sec. Note the result.
Experiment by changing input numbers one by one. For, instance, try
the initial vz = 5 ft/sec. Increase the starting speed.

   A fast ball might have vx = 120 ft/sec, or higher. If this has
an initial negative value of vz, then the bounce may occur early and
the ball may rise to imperil the batsman; this is a 'bouncer.' Spin
is not important here; the key is intimidation!

   Now in $<$Enter Parameters$>$ select $<$Swing$>$. There is no spin, but
a sideways force results from an asymmetric orientation of the ball's
seam (raised stiches round the 'equator' of the ball). This orientation
is controlled by the parameter F as explained in the input screen
for the starting conditions.

   Enter vx = 100, vy = --4.5, vz = -- 5 ft/sec. Use the default value
F = 0.4. Observe the motion. Now vary the conditions to see the
variety that can be introduced.


\subsection{Shuttlecock Trajectories in Badminton:}

   The program opens with an input screen for the starting conditions.
Choose the default values, with the hit at height 8 ft, --15 ft from
the net, with speed 100 ft/sec making the angle 20 degrees with the
horizontal. Note the extreme effects of drag on the trajectory.

   Now change the initial angle from 20 to 30 degrees and notice that
the range is hardly affected.

   Repeat, starting at a height of 6 ft.

   Use this as a basis for experimentation. You can execute a variety
of different shots, and services. For some details of this variey
see the text for a partial classification.


\subsection{The Motion of a Discus}

   One of the interesting features of this model is that the range
of a discus can be increased if it is thrown into the wind. To
confirm this, start by selecting $<$Run$>$, $<$Enter parameters$>$ from
the menu. Accept the default values with the exception of the
wind speed: enter 20 m/sec. Then select $<$Throw the discus$>$ and
see the trajectory noting the range. Next press the hotkey $<$Input$>$.
Change the  wind speed to 10. Finally, repeat with wind speed zero.
(You must start with the greatest range in order to have all
trajectories fully visible.)

   Now vary other parameters. Experiment with different angles for
the throw. What might you do to maximize range if the wind is behind you?


\subsection{The Motion of a Javelin}

   Select $<$Run$>$, $<$Enter parameters$>$ from the menu to enter the
starting conditions for a throw. For a start, use the default values.
Look at the trajectory.

   Next, press the hotkey $<$Input$>$ and experiment with the starting
conditions, changing them by small amounts, one at a time. In particular,
find the effect of the starting angle on the range.

   The parameter concerning the javelin itself is the distance of the
center of pressure from the center of mass. Look at the effects of
small changes in this quantity.



\section{Space Flight and Astronomy}


\subsection{A Trip to the Moon}

   The menu item $<$Run$>$, $<$Starting conditions$>$ prompts you to enter
the conditions for launch from a parking orbit around the Earth. $<$See
the animation$>$ shows the animated orbital motion. The second usually
follows automatically after the entry of the starting conditions.

   In the starting screen note the interpretation of the start. Accept
all the default values and then see the motion: it should result in a
direct hit on the Moon. Become familiar with the options  for marking
the positions in the orbits of the spacecraft and the Moon.

   Make the position at launch from the parking orbit the parameter
to vary. Initially it is --40 deg. Try -- 39 deg. You should miss the
Moon and return to the Earth. Next try --41, --42 and then --43 deg.
Ultimately, you should escape from the Earth--Moon system.

   If you start with conditions chosen at random, you are very
unlikely to approach anywhere near to the Moon. So change things
gradually. Experiment with changes in the speed at launch, or the
size of the parking orbit.

   You will find situations when the orbit of the spacecraft undergoes
big changes before, maybe, hitting the Moon. Note that you can
change the scale to enable you to follow motion of this sort.


\subsection{The Descent of Skylab}

   The model is very sensitive to the starting conditions. You may
find yourself circling the Earth endlessly without apparent change
in the shape of the path, or you may crash at once. Try to find
situations where you have six to ten revolutions before landing.

   The menu item $<$Run$>$, $<$Paraneters and initial conditions$>$ is used
for starting any orbit. It is simplest to start, as suggested,
on the x--axis, moving perpendicular to the x--axis; then just two
numbers are needed for the initial conditions. Take the default value
for x, and put vy = 7.85. Keep all other default values, but note the
interpretations.

   Before starting the animation, press the hotkey $<$Notrail$>$ so that
$<$Trail$>$ appears. Then press $<$Run$>$. You should see several revolutions
before landing.

   Look at the way the orbit changed. The closest point (perigee)
remained about the same distance from the Earth, but the furthest
point (apogee) approached the Earth. Most of the drag is near perigee,
where the atmospheric density is greatest. Eventually, the orbit is
nearly circular, drag is strong all the way around, and the decay is rapid.

   Another way to see this effect is to plot the altitude (the distance
from the surface of the Earth) as a function of the time.

   Now diminish the intitial vy by small amounts until skylab has just
one revolution before landing. Then increase it to see many revolutions.

   Try starting with smaller initial altitude (so the initial drag is
much greater).

   You can vary three quantities: the crosssectional area and the mass
of Skylab and the value of the drag coefficient. The actual effect of
drag on the acceleration is proportional to

           (drag coefficient)$*$(Area)/Mass

\noindent so you can effectively change the drag by altering just one of these
three quantities.


\subsection{The Range of an ICBM}

   Select $<$Run$>$, $<$Parameters$>$ from the Menu. Note the choices. These
include the altitude of launch and a factor for exaggerating the
scale of the latitude. Also take note of the geometry of launch that
is shown to the left. Keep the default values for now.

   On the main graph you will see sliders that can be used to vary
the speed at launch and the launch angle. With the default values,
press $<$Fire$>$. Now press $<$Aim$>$ and change only the angle; try to
maximize the range.

   Change the launch speed. Again find the angle that maximizes the range.

   In experiments, vary the launch height and the cross--sectional area
and mass of the ICBM. Increasing these will diminish (mass) or increase
(area) the drag on the ICBM.


\subsection{Jupiter and a Comet}

   Select $<$Run$>$, $<$Parameters for Jupiter$>$ from the menu. These are
set for the actual values. Later, you may find it interesting to vary
them. (You might pick values for Saturn, for instance).

   $<$Initial conditions for the comet$>$ enables you to start the motion.
Study the geometry and the definitions of the units. The initial default
value of the speed is that that IF you accept the default values, the
closest distance of the comet from the Sun in the initial revolution
will equal the radius of Jupiter's orbit. If you change the initial
distance, and still want the same closest distance, use the button to
make that selection. For now, keep the default values, but take the
speed to be 0.0308 units.

   In the graphics screen press the hotkey $<$Run$>$. (You may want to
slow down the animation.) The comet will be 'captured' into a much
smaller orbit, with aphelion distance (the furthest distance from
the Sun) close to the radius of Jupiter's orbit. Any close encounter
between Jupiter and the comet is shown in a window to the right.

   I recommend that you let the motion keep running. There are many
further close encounters, and many changes in the shape of the orbit.
If part of the orbit leaves the screen, use the $<$Rescale$>$ option, and
choose to continue the present orbit.

   Next, return to the menu and the input screen for initial conditions.
Change the starting angle of Jupiter to 186 deg. This will result is the
immediate expulsion of the comet from the solar system.



\subsection{Motion Close to L4}

   It is easy in the program to produce nonsense and to get lost,
especially when dealing with Poincare maps. For a start, avoid these.

   It is convenient to have names for the two attracting bodies;
they are called the Earth and Moon, although the mass parameter need not
apply to them. This parameter is:

      (Mass of Moon)/(Sum of the two masses).

The input screen for this will appear when the program starts; it can
also be accessed by selecting $<$Run$>$, $<$Enter mass parameter$>$ from
the menu. Choose the default value 0.01.

   You will need to get used to the units of the model. The constant
distance between the Earth and the Moon is one unit of length. The
period of the Moon's orbit around the Earth is 2*Pi units of time.

   There are two ways to view the orbit of the third body. In a
fixed reference system we see the Earth at the center, the Moon, M,
moving around  the Earth, the point L4 (one unit of length from the
Earth and the Moon) also moving around the Earth, and, finally, the
third body, hopefully, moving around L4. Let's view this first.

   Select $<$See orbits with fixed axes$>$. You are prompted for initial
conditions relative to L4. Use the default values: (0.02, 0) and (0, 0).

   The resulting animation can be confusing when it is the motion
relative to L4 which is of concern. The other option is to view motion
from a rotating reference system in which the Earth, the Moon and L4
are all fixed. Select $<$See orbits with rotating axes$>$. Now the motion
appears more simply.

   Now repeat, changing only the initial value of the dx, the
x--displacement from L4. 0.01 and 0.02 result in motion that remains close
to L4; 0.03 results in escape. Try 0.025, 0.026,... You should find that
the borderline between bounded motion and escape is between 0.028 and 0.029.

   Repeat the experiment using different values of the mass parameter.
Some special values are given on the input screen. Confirm the
properties of these values, as stated.

   Now prepare to see some Poincare maps. Take 0.01 for the mass
parameter and see the orbit with start (0.01, 0), (0,0). You will see at
the top of the screen the statement:

    Energy for these conditions: C = 3.0000759.

C is a constant for this orbit.

   The Poincare maps are to compute different oribts having this same
value. Select $<$See Poincare maps$>$. Accept the value of C. Choose to see
the orbits in addition to the maps. Choose to enter initial conditions
from the keyboard. The first orbit should have the same conditions used
just before. Let this run until you are seeing a definite pattern.

   Press $<$NewMap$>$. Replace dx = 0.01 by dx = 0.011. Continue, gradually
increasing dx until the orbit becomes unbounded. If the pattern of the
map extends beyond the screen, use the $<$ReScale$>$ option.

   For a Poincare map C is given, and initial conditions dx, vx are
input. Initially dy = 0 and vy is found, consistent with the value of
C. Motion is followed in the three dimensional space dx -- dy -- vx.
Whenever dy = 0, the point (dx, vx) is plotted in the dx -- vx plane.

   Continuing the experiment, see the map with starting conditions
(0.014, 0). Note the shape built up by the points. Now select $<$See
Poincare maps in 3--D$>$. Accept default values, which will refer to
the computation just made. You will see the motion in three
dimensions, with points marked where the orbit crosses the x -- vx
plane. In three dimensions the orbit lies on a torus. This can be
hard to view. Try changing the direction of viewing so that you can
see the hole in the torus.



\subsection{Aerobraking a Space--craft}

   The parameters of the model are those that define the target
planet and its atmosphere, and the incoming spacecraft. For a start,
accept the default values for these and for the initial conditions.
The graphics screen will show a non--circular planet; this is due to
the scale. See the orbit, which will result in atmospheric braking.
Press $<$CleanUp$>$ and change the minimum value of y to --100000. Try again!

   Small changes in the starting conditions make a big difference.
So it is best to use the keyboard for experimenting. Find ranges
of values for the start that will lead to capture.

   Change parameters one at a time. For instance, you might start by
   varying the planetary atmosphere.


\section{Pendulums}


\subsection{The Simple Pendulum}

   The menu item $<$Data$>$, $<$Enter parameters$>$ will appear automatically
at the start of the program. To start, accept the default input, so
that there will be no friction.

   Next select $<$Animate$>$, $<$Swing the pendulum$>$. Follow the prompts to
use the sliders to choose initial conditions, and observe the animation.
Then select $<$Time plots$>$ to see the angular displacement and angular
velocity plotted as functions of the time.

   Finally, select $<$Phase-plane$>$, in which the axes are angular displacement
(horizontal) and angular velocity (vertical). Accept the default quantities
on the input screen. Use the mouse to choose initial conditions, and
follow the animation to interpret the closed and open curves in terms of
oscillation and circulation of the pendulum.

   Note that the phase-plane axes are set so that circulatory motion 
(which would normally continue indefinitely in the horizontal sense)
leaves the screen, but is then continued from the opposite end.

   Now return to $<$Data$>$, $<$Enter parameters$>$ and enter a small number,
say, 0.1, for the friction parameter, choosing regular friction. Repeat
observations, using each of the three utilities. Then increase the
friction parameter.

   Finally, select dry friction, and see if that differs significantly
from regular friction in the phase-plane.



\subsection{A Magnetic Pendulum}

   The menu item $<$Data$>$, $<$Enter parameters$>$ will appear automatically
at the start of the program. To start, accept most default input numbers,
but set the friction parameter equal to zero.

   Next, select $<$Animate$>$, $<$Swing the pendulum$>$. Follow the prompts to
use the sliders to choose initial conditions, and observe the animation.

   Next, select $<$Time plots$>$ to see the angular displacement and angular
velocity plotted as functions of the time.

   Finally, select $<$Phase-plane$>$, in which the axes are angular displacement
(horizontal) and angular velocity (vertical). Accept the default quantities
on the input screen. Use the mouse to choose initial conditions, and
follow the animation to interpret the closed and open curves in terms of
oscillation and circulation of the pendulum.

   Return to $<$Data$>$, $<$Enter parameters$>$. Enter a small number, say 0.1,
for the friction parameter. Now follow the motion as before. See if you
can guess in advance which of the two stationary magnets will finally
hold the swinging magnet. in $<$Time plots$>$ select  time interval large
enough for the motion to followed until one of the stationary magnets
holds the pendulum.

   In the cases just run the system had four equilibria, two stable
(pointing, approximately at a fixed magnet) and two unstable (vertical,
up or down). This can be demonstrated in the utility $<$Data$>$, $<$See
equilibria$>$. Select this. You will see a plot of the torque acting on the
pendulum as a function of angular displacement. At an equilibrium this
torque is zero. The pendulum is plotted in the positions of equilibrium.

   You can now use sliders to experiment with different values of the
parameters, looking for values that can lead to more interesting motion.
For instance, change H to be --1, so that the point of suspension of the
pendulum is below the fixed magnets. You have 3 stable and 3 unstable
equilibria. Have a look at the resulting motion.

   Now experiment with changing the strengths and polarities of the
fixed magnets, (C1 and C2) as well as their displacements from the
center (A1 and A2).



\subsection{A Child on a Swing}

   The menu item $<$Run$>$, $<$Parameters$>$ will appear at the start of the
program. You can specify dimensions of the swing and the child, the mass
of the child and a parameter for the frictional resistance. (The resisting
acceleration is proportional to this parameter divided by the mass.) You
can also specify how to control the animation: let the program (or the
child) decide, or do it yourself, with a hotkey.

   For a start, accept default values, and let the motion be controlled
automatically. Also accept the default starting conditions (i.e., give
the kid a push, at the lowest point) and watch the animation.

   You may find it helpful to select $<$Phase-plane$>$ from the menu or in
the initial input screen. Try it.

   Next, if you have followed the technique of the child, select
$<$Parameters$>$ and manual control. Any good?

   With the default values of the parameters the motion has increasing
amplitude until circulation starts. If the friction is inreased (or
the mass decreased) you should be able to establish limit cycles. Look
for values of the parameters that will achieve this.



\subsection{A Spring Pendulum}

   A the start of this program an input screen appears. (This can
later be accessed through $<$Run$>$, $<$Parameters$>$ in the menu.) The
quantities to be entered include the initial conditions and the
spring constant. Accept all default values. Then you will see
the motion in animation. Enjoy it.

   Next select $<$See Poincare maps$>$. First, accept all default values
in the input screen. Follow the prompts in choosing a color and then a
starting point. Then press $<$Start$>$. You will see the pendulum in
animation. Whenever it is vertical, the values of its length, L, and
the rate of change of this length, Ldot, are calculated and a point
plotted on the screen, which is the L--Ldot plane.

   Let the program run until you see a pattern forming. Then press
$<$Input$>$ and choose an initial position further from the center.
Continue experimenting. You should eventually see chaotic structure
and animation.

   Seeing the animation slows down the computation of the maps. So
select $<$See Poicare maps$>$ once more, but choose not to see the animation.
Follow the same procedures as before. Build up a detailed picture of the
chaotic and non--chaotic structures.

   In the dynamical model there is a relation of the form
          f(L, Ldot, Theta, Thetadot) = h.
h is a constant for any solution. All solutions followed in the L--Ldot
plane have the same h. The value of h is specified in the input screen
for the Poincare maps.

   Experiment further using different values of h.



\subsection{The Double Pendulum}

   At the start of this program an input screen appears. (This can
later be accessed through $<$Run$>$, $<$Enter initial conditions$>$ in the menu.)
The quantities to be entered are the initial conditions for the two
pendulums. Accept all default values. Then select $<$Animate the pendulums$>$.
You will see the motion in animation. Enjoy it.

   Next select $<$See Poincare maps$>$. First, accept all default values in
the input screen. Follow the prompts in choosing a color and then a
starting point. Then press $<$Start$>$. You will see the pendulums in
animation. Whenever they form a straight line the values of Theta and
Thetadot are plotted as a point in the cos(Theta)--Thetadot plane. (Theta is
the angular displacement of the upper pendulum.) Let the program run
until you see a pattern forming. Then press $<$Input$>$ and choose another
initial position. Continue experimenting. You should eventually see
chaotic structure and animation.

   Seeing the animation slows down the computation of the maps. So
select $<$See Poicare maps$>$ once more, but choose not to see the
animation. Follow the same procedures as before. Build up a detailed
picture of the chaotic and non--chaotic structures.

   In the dynamical model there is a relation of the form                                    14
            f(Theta, Thetadot, Phi, Phidot) = Energy.
Energy is a constant for any solution. All solutions followed in the
cos(Theta)--Thetadot plane have the same Energy. The value of h is specified
in the input screen for the Poincare maps.

   Experiment further using different values of Energy.



\subsection{The Dumbbell Satellite}

   The satellite moves in an elliptic orbit in accordance with Kepler's
laws. For a start, I suggest that you follow the demonstration in $<$Help$>$,
$<$Elliptic orbit demo$>$ from the menu. Become familiar with the way the
eccentricity affects the shape of the orbit and the kinematics.

   The principal input screen will appear automatically at the start
of the program. It can be accessed through $<$Run$>$, $<$Enter parameters$>$.
The default eccentricity is 0; leave it. The motion will start at
pericenter. For initial conditions you can enter the initial angular
deflection of the line of masses from the major axis (Theta) and the
rate of change of this angle (ThetaDot). The period of the elliptic
orbit is 2*Pi units of time; so if you enter 360/2*Pi = 57.29578 for
ThetaDot, the initial rotation rate of the pendulum will be equal to
the average angular orbital rate.

   Accept all intial default values. You will then automatically see
the animated motion of the satellite.

   The satellite is a 'pendulum' in orbit; the lowest point of the
pendulum being equivalent to the line of masses pointing to the center
of the Earth. The angle the line makes with this direction is called
Phi in the program. Select $<$Run$>$, $<$Enter parameters$>$. Enter 40 degrees
for the initial displacement, and look at the animation. The pendulum
analogy may be clearer.

   Phi and its derivative, PhiDot, are the coordinates in the phase-plane.
Select $<$Phase-plane$>$. Press $<$Input$>$ to enter starting conditions. Use
the mouse. Click on a position close to the center and see the plot in the
phase-plane and the animated oscillations. Now pick starting points
further away, until you observe circulation. The parallel with the
simple pendulum should be clear.

   Select $<$Run$>$, $<$Enter parameters$>$. Enter 0.1 for the eccentricity and
0 for the initial deflection. The animated motion is more interesting;
but on average, the pendulum seems to be pointing toward the Earth.
Next change the initial deflection to 40. Observe the motion. Now change
it to 50; the motion has now become chaotic.

   Let's follow this in the phase-plane. This time, use the keyboard to
enter starting conditions. Enter (40, 0) for the the initial (Phi, PhiDot).
(PhiDot = 0 is equivalent to ThetaDot = 360/2*Pi.) The plot is tangled and
hard to interpret. Look also at motion with starting conditions (0, 150)
and (50, 0). Interpretation becomes simpler if Poincare maps are followed.

   Select $<$Poincare map$>$. Choose to see the animation. Use the keyboard
for starting conditions. Make three runs with conditions (40, 0), (0, 150)
and (50, 0). Follow each run until you see a pattern forming. The program
is following the calculation for the phase-plane plot, but only plots
points at times 0, 2*Pi, 4*Pi, 6*Pi,..., when the satellite completes
an orbital revolution.

   Press $<$CleanUp$>$. This time, choose not to see the animation. Repeat
the runs. You will see better plots, more quickly.

   Now try runs successively with conditions (0, 140), (0, 139),
(0, 138), (0, 137). Having seen the figures, press $<$CleanUp$>$ and choose
to see the animation. Repeat the last run, noting what is taking place
in the animation. The satellite completes two rotations of Phi
for every orbital revolution.

   Select $<$Plot Phi vs time$>$. Change the range of time to the plot to 50.
Keep the starting conditions (0, 137). Go to the following input screen.
Interpretation becomes harder! In inertial space, the satellite rotates
three times for every single revolution. Then the angle (V -- 3*Theta)
will oscillate. Enter the numbers 1 and 3. Now see the plot.

   This is an introduction to the utilities. If you increase the
eccentricity the chaos will become more pronounced. Two parameters
have been ignored. You can include a coefficient for a resisting term.
This might apply to a satellite with a de--spinning torque.

   The equations for the model can be modified to apply to a non--spherical
satellite if the term for the acceleration of Theta is multiplied by a
factor less than one. See the second input screen for further details.



\section{Miscellaneous}


\subsection{Production and Exchange}


   Select $<$Run$>$, $<$Enter parameters$>$ from the menu. There are six
parameters; read and remember their interpretation. Accept the default
values. The dependent variables are x and y, representing the goods
owned by the two men in the model. Accept the default values for the
scale of the plot.

   You will see a graphics screen, with coordinates x and y and an
equilibrium position at the point (1,1). There are six sliders that
enable you to change the values of the parameters. Experiment with
these, making small changes, with interpretation, and see the effects
on the equilibrium. Then press $<$Input$>$, choose the mouse, and click on a
starting point. As the solution procedes, you can change the parameters.
Try to keep the equilibrium on the screen. If it disappears, one man will
cease to work, becoming a parasite.



\subsection{The Economics of Fishing}

   Select $<$Run$>$, $<$Enter parameters$>$ from the menu. There are four
parameters. a and b deal with the logistic growth of the fish; the
maximum sustainable population is (a/b). Remember that numbers are
scaled: a population of '2' might stand for 2 million. C represents the
cost of fishing and P the price the catch will fetch at the market.
Accept the default values. Select the mouse for choosing starting
values. Do not choose periodic variation in the fish and cost parameters.

   Now select $<$Plot solutions$>$. You will see a graphics screen with the
fish population as the horizontal variable and E, the effort put
into fishing, as the vertical variable. Equilibrium positions are marked.
(The origin is always an equilibrium; this is not marked.) You can use
the sliders to vary the equilibrium that is not on the horizontal axis.
Experiment with this, interpreting the changes.

   Press $<$Input$>$. Choose a starting point. (Don't start at an
equilibrium!) While the motion is taking place, experiment with
changes of the parameters.

   You can allow the fish birth rate, a, or cost of fishing, C to vary
periodically. In this case there are no equilibria. Experiment
first with small changes.

      Try not to wipe out the fish.


\subsection{The Motion of a Yoyo}

   The main input screen will appear at the start of the program.
This can also be accessed by selecting $<$Run$>$, $<$Enter parameters$>$
from the menu. Become familiar with the interpretation of these
parameters, in the design of the yoyo. Accept the default values.
Then accept the starting speed. Watch the animation. This can be
repeated by pressing $<$Yoyo$>$.

   You can change the parameters and the starting speed by pressing
hotkeys. The object is to maximize the angular velocity at the
lowest point. Work principally on the outer radius of the central
sprindle (around which the string is wound), the radius of the yoyo
and its moment of inertia.



\subsection{The Action Between a Violin Bow and String}

   The input screen for parameters and initial conditions appears at
the start. This can also be accessed by selecting $<$Action$>$, $<$Enter
Parameters$>$ from the menu. Accept the default values. In the graphics
screen press $<$Start$>$ and watch the animation. If you want to change
the length of time for the action, press $<$Re--scale$>$.

   Now increase the friction by raising the value of A; this is the
maximum frictional force when there is no relative velocity between
the surfaces. (B is the maximum when this velocity is nonzero.) Follow
the effects. Then investigate the effects of changing B.

   Next experiment with changes in the speed. Finally, make adjustments
to the spring constant.



\subsection{Landing an Aircraft on an Aircraft Carrier}

   Select $<$Help$>$, $<$Run demonstration$>$ from the menu. You will learn
about the different types of motion that are possible.

   Next select $<$Run$>$, $<$Enter parameters$>$. Look through the list; each
one can be varied in experiments. Accept the default values. Then
in the graphics screen press $<$Land$>$.

   The different types of motion are shown. The animation is not true
since the numerical calculations are greater for some phases. The
aircraft should land safely.

   Press $<$Input$>$ and change the landing speed. See if you can just land
the aricraft on the edge of the deck!

   Now return to the input screen and experiment. Change one parameter
at a time.


\subsection{Pitching and Rolling at Sea}

   Start by selecting $<$Help$>$, $<$About the model$>$ from the menu.
Become familiar with the parameters. Then select $<$Run$>$, $<$Enter
parameters$>$. Look at the numerical values; accept the default values.

   Select $<$Animate$>$. Notice the initial deflections marked on the
sliders. (The initial rates of change are zero.) Press $<$Run$>$.
Although there is no initial rolling, notice the way that rolling starts.

Another way to see the motion is to plot the angles as functions of
the time. Select $<$Time plots$>$. The default angles should be those just
used; make sure that the angular rates are zero.

   See how great you can make the initial values of pitching and
rolling without disaster.

   Parameters of importance are the natural frequencies of pitching
and rolling, and the frequency of the sea. The default values are,
approximately, 1, 0.5 and 0.5; so there are possible resonances
between them, leading to large transfers of energy between the the
modes of oscillation. See what happens if you use numbers such as
1, 0.7, 0.6.

   There are two parameters that control the coupling between pitching
and rolling. Experiment with increasing these.

   Also experiment with the resisting terms. Suppose that the ship's
stabilizers fail, so that these become zero. Increase the forcing
amplitudes, so that the sea becomes rougher.



\subsection{Motion of a Ball in a Rotating Circular Ring}

   The menu item $<$Run$>$, $<$Enter parameters$>$ enables you to enter
values for the angular rate of the spinning hoop and a parameter for
the friction between the ball and the hoop. This must be used before s
electing $<$See phase-plane$>$. If you select $<$Animate$>$, then these
two parameters, together with the initial conditions for the ball,
can be controlled using sliders.

Select $<$Animate$>$. Start with default values. Then increase the initial
displacement of the ball. You will observe pendulum--like motion.

   Increase the rotation rate of the hoop. When it is large enough,
the lowest point of the ball will become unstable, and two points
of stability will appear to the sides. This is and example of 'bifurcation.'

   Repeat this experiment introducing nonzero values for k, the
friction coefficient.

   Choose values of the spin rate for which the lowest point of the ball
is stable, and set the friction coefficient equal to zero. Then select
$<$See phase-plane$>$. Build the phase-plane plot, showing different types of
motion.

   Now increase the spin rate until the  bifurcation occurs.

   Repeat, with nonzero values for the friction coefficient.



\subsection{The Swinging Atwood Machine}

   The main input screen will appear at the start of the program.
This can also be accessed by selecting $<$Run$>$, $<$Enter parameters$>$
from the menu. Accept the default values. Then select $<$Animate the
machine$>$, and follow animated motion. When you get tired of that,
select $<$Plot path of pendulum$>$ to see the pattern made by the end
of the swinging pendulum.

   Repeat with more adventurous initial conditions; try an initial
displacement of 90 deg.

   Now select $<$See Poincare maps$>$. To start, choose to see the animation
along with the plotting of the maps. Follow the prompts, and choose a
sufficient variety of starting conditions to establish the patterns of
the maps. Any time you become impatient with the slow progress, due to
the animation, press $<$CleanUp$>$ and choose not to see the animation.
                                                                                                   

   If you see some interesting map and would like to see the animation,
then select $<$Animate the machine$>$. The starting conditions will be the
continuation of the motion that you were following.

   The default value of 3 for the ratio of the masses leads to non--chaotic
motion. Try 1.1. When you play with Poincare maps, you will see some
organization in the central part of the screen and some toward the edge
of the parabolic region where the mapping takes place. You will also
find that there are conditions leading to the length of the swinging
pendulum becoming zero. When this happens, the calculation stops.

   Change the mass ratio by small amounts, to see how this affects the
non--chaotic and chaotic behavior.



\subsection{A Chaotically Driven Wheel}

   This program has many similarities to that of the Forced Pendulum.

   Start by selecting $<$Run$>$, $<$Enter parameters$>$ from the menu. Look at
them and note their interpretations, but accept the default values.
Similarly, select $<$Enter initial conditions$>$, and accept the default
values. Then select $<$Animate$>$.

   For the animation you can change the values of the parameters and
initial conditions, using the hotkeys. But for now, stay with the
default set.

   When you have tired of the animation, look at the time plots.
Plotting Phi (the angle for the orientation of the forced wheel) will
not be effective here, since Phi is increasing indefinitely. Have a
look at a plot of sin(Phi). Select $<$Plot Sin(Phi) versus time$>$.

   Now select $<$Run$>$, $<$Plot Phi-prime versus time$>$. Using the default
values for the scale, you will get a good idea of the nature of the
motion.

   Next, select $<$PhasePlot$>$, $<$See phase-plane plot$>$. For the lower l
imit of Phi-prime enter 0, and for the upper limit enter 1000. You may
want to see the plot only after an initial period of transient motion.
For the moment, enter 0 for the time delay. Next, press $<$Keyboard$>$ to
enter the starting conditions from the keyboard. Accept the default values.
Enjoy the plot and the animation.

   If the plot becomes too messy, press $<$CleanUp$>$ and choose the option
to continue the present plot. The existing plot will be erased, and you
can continue to plot where the motion left off.

   For another way to avoid the plot of the early transient motion
press $<$Time delay$>$, enter 10 and then press $<$CleanUp$>$. This time,
choose a new plot. Use the keyboard once more to enter the starting
conditions. You will see the early animation, but not the phase-plane
plot. If this becomes too boring, press a key to stop.

   The delay option is principally for use in plotting Poincare maps:
i.e., sampling the phase-plane plot once in every period of the
forcing term. Select $<$See Poincare map$>$. Use the same scale as before,
and enter a delay parameter of 15. Use the keyboard to enter the default
starting conditions. The points that appear, show the final transient
motion that approaches the final periodic state. This final state is
represented by a dot. The output on the screen will becme more
concentrated if the delay parameter is increased.

   Finally, look at a bifurcation diagram. Select $<$See bifurcation
diagram$>$ and examine the first input screen. The parameter that is
changed on the horizontal axis is k/I, where k is the spring constant
and I is the moment of inertial of the driven wheel. You are
prompted to enter the extreme values and the number of sample values
to be used. Few values will produce a faster plot, but with less detail.

   Values of Phi-prime are plotted on the vertical scale. For a sharper
diagram, the initial, transient motion should be ignored. The first p
periods of the forcing term will be ignored. After that, q points are
plotted. Higher values of p and q may give better plots, but the plotting
will take longer.

   Finally, you can enter the starting value of Phi. This will
affect the diagram.

\end{document}

